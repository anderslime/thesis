\chapter{Abstract}

% Scope + Problem + Existing
Caching is a popular solution for improving the performance and scalability of web applications. When data have been cached the web application does not have to recompute the data until it has been invalidated. Unfortunately, these caching systems presents two major challenges. First, they leave the responsibility of locating the cached values, invalidating correctly, and keep the values up to date, which are tasks that often leads to additional application complexity and programming errors. Secondly, in highly dynamic web applications, where the cached values are invalidated frequently, the cache hit rate can become low such that more users must wait for the cached values to be recomputed. This can be critical for the user experience in cases where the computations are slow.
% This solution

This thesis address these challenges by introducing a new cache called Smache. Smache uses a programming model that lets the programmer mark a function to be cachable by declaring the dependencies to the underlying data. The cached function is then automatically invalidated and afterwards updated when the underlying data changes. Smache uses timestamp invalidation to be able to update all cached values concurrently and thereby allowing to improve the update throughput by adding more processes to update cached values.
% Results

This was verified by our experiments that showed Smache is able to improve the throughput of updating cached values by adding additional processes. Furthermore they showed that Smache only introduces a constant performance overhead of $2\text{ }ms$ to application operations updating data in the primary storage.

