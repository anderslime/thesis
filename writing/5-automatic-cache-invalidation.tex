\chapter{Automatic Cache Invalidation}
\label{chapter:invalidation}

In the model for cachable functions described in the previous chapter~\ref{chapter:smache-cachable-functions} we've only described how the system can make a function cachable such that Smache automatically stores and locates the cached object returned by the function. While this removes the burden of naming and localizing cached objects, it does not solve the hard problem of invalidation.

To make cache invalidation easier with Smache, we will extend this solution with an invalidation mechanism that automatically invalidates the cached objects based on dependencies declared by the programmer.

This chapter describes how the suggested system uses the declared dependencies to track updates to underlying data and invalidate cached objects correctly. The solution for automatic invalidation is described in three parts. First we introduce the data structures used to achieve fast and efficient invalidation (section~\ref{sec:simple-object-dependence-graph}). Second, the \emph{dependency registration} describes how the cached objects are registered into the ODG. Finally, the \emph{invalidation propagation algorithm} (section~\ref{sec:invalidation-propagation} describes how Smache invalidates affected cached objects when changes are made to the underlying data. The chapter ends with a discussion on how the suggested solution meets the criteria and requirements of the thesis.

% section dependency-registration end

\section{Simple Object Dependence Graph}
\label{sec:simple-object-dependence-graph}

The Object Dependence Graph (ODG) was first described in a series of papers by IBM~\cite{paper:ibm, paper:ibm-extended} and was build to support the content management system build for the Olympic Games in 2000. The final content served to the user is build from HTML-fragments and HTML-pages editable by the users as well as underlying data periodically changing. To be able to serve the final documents fast, the system pre-generates the HTML pages such that the system doesn't have to generate the pages on each request. To do this efficiently, the system maintains dependencies between the different kinds of objects and updates depending objects when underlying data changes. The cache object pre-generation can also be described in terms of caching, where the system uses an in-place caching approach.

Jim Challenger et.al.~\cite{paper:ibm-extended} presents a simple and a generalized version of the ODG. The generalized version is described for a content management system and includes a number of enhancements compared to the simple model that is unnecessary for the purpose of this thesis. We will therefore use the simple ODG that is described as following:

ODG can be represented using a directed graph, where a vertex either represents underlying data or an object that is a pure transformation of underlying data. An edge from a vertex representing underlying data $u$ to a vertex representing an object $o$ denoted ($u$, $o$) indicates that a change to $u$ also affects $o$.~\cite{paper:ibm-extended} gives the following constraints for the simple ODG:

\begin{itemize}
  \item Each vertex representing underlying data does not have an incoming edge
  \item Each vertex representing an object does not have an outgoing edge
  \item All Vertices in the graph correspond to underlying data (nodes with incoming edges) or objects (nodes with an outgoing edge)
  \item None of the edges have weights associated with them
\end{itemize}

Figure~\ref{fig:simple-odg} illustrates an instance of a simple ODG with four vertices of underlying data ($u_1$, $u_2$, $u_3$ and $u_4$), two vertices representing objects ($o_1$ and $o_2$), and the dependencies between them. Where we can see that when underlying data $u_2$ changes, the system must update both the cached object of $o_1$ and $o_2$ and when $u_4$ changes, it only needs to update $o_2$.

\begin{figure*}[ht!]
  \centering
  \includegraphics[width=0.6\linewidth]{figures/simple-odg.pdf}
  \caption{An example instance of a simple ODG}
  \label{fig:simple-odg}
\end{figure*}

% section simple-object-dependence-graph end

\section{Dependency Data Structure for Cachable Functions}
\label{sec:dependency-data-structure-for-cachable-functions}

% TODO: Describe this in a way so that we understand the purpose of the graphs instead of how they are represented
% TODO: Make it clear that we only use these data structures to represent the dependencies

There exists two representations of dependencies for cachable functions. The first one is the static representation, which can be derived directly from the declarations in the source code. This representation only describes dependencies between the declarations and not cached object instances. We will define this representation as the \emph{Declaration Dependence Graph (DDG)}. The other representation is dynamic and describes the dependencies between the entities of the underlying data and the cached object instances. We define this as the \emph{Instance Dependence Graph (IDG)}.

\subsection{Declaration Dependence Graph}
\label{subsec:declaration-dependence-graph}

When the cachable functions are defined as in section~\ref{subsec:making-functions-cachable}, the dependencies are declared using the entity types. The DDG is an extension of the Simple Object Dependence Graph, where the cached functions corresponds to the object vertices and the entity types corresponds to the underlying data. Where the simple ODG only includes direct dependencies, the DDG has two kind of dependencies. An edge from $u$ to $o$ denoted $(u, o)_d$ represents a direct dependency, which means that the dependency defines the given $o$. That is if a given cached object instance has a dependency to an instance of an entity then it cannot change. The direct dependencies includes an index indicating a position such that there is an order of direct dependencies for a given cached object. An edge from $u$ to $o$ denoted $(u, o)_l$ represents a lazy dependency, which means that the dependency can change throughout the lifetime of the cached object instance and has to be evaluated before invalidation to derive the dependency for the instance. This could just as well be two graphs, but we model it as one for illustrative purposes.

Figure~\ref{fig:declaration-dependence-graph} shows the DDG for the running example from code snippet~\ref{code:running-example}. For example we see that the cached object for the \verb$participant_score$-function depends directly on the participant and it depends lazily on the grade. This is because there exists a participant for each score, but the dependencies from a cached object instance to any grade entity can change throughout the lifetime.

\begin{figure*}[ht!]
  \centering
  \includegraphics[width=0.6\linewidth]{figures/declaration-dependence-graph.pdf}
  \caption{The Declaration Dependence Graph of the running example}
  \label{fig:declaration-dependence-graph}
\end{figure*}

The DDG graph is represented using two data structures. The lazy dependencies are represented using an outgoing adjacency list from the underlying data nodes. The direct dependencies are represented using an incoming adjacency list from the object nodes ordered by the position index of the dependency. To access a given adjacency list fast we use hash tables index by entity type for the outgoing adjacency list and by the object ID for the incoming adjacency lists as depicted on figure~\ref{fig:ddg-data-structure}.

\begin{figure*}[ht!]
  \centering
  \includegraphics[width=0.6\linewidth]{figures/ddg-data-structure.pdf}
  \caption{An illustration of the data structure representing the DDG on figure~\ref{fig:declaration-dependence-graph}}
  \label{fig:ddg-data-structure}
\end{figure*}

New dependencies can be added to the DDG using

\begin{itemize}
  \item \verb$add_dependency(s_id, t_id)$: Adds a dependency between the source id \verb$s_id$ used as index for the hash table and the target id \verb$t_id$ used as element in the adjacency list.
\end{itemize}

When \verb$add_dependency$ is used we check if there already exists an adjacency list in the hash table using \verb$s_id$. If there exists one, we add \verb$t_id$ to the given adjacency list, or else we add a new adjacency list only including \verb$t_id$ and indexed by \verb$s_id$.

The DDG is build by iterating through all the cached functions with their respective dependencies and add the dependencies to the data structure as described above. The lazy dependencies are added with the entity type as \verb$s_id$ and the cached function as \verb$t_id$. The direct dependencies are added using the cached function as \verb$s_id$ and the entity type as \verb$s_id$.  The advantage of the DDG is that it does not change in the lifetime of the application, which means it can be preprocessed when the application starts.

The queries needed for automatic cache invalidation are the following:

\begin{itemize}
  \item \verb$lookup_lazy_dependency(entity_type)$: Finds all lazy dependencies from a given entity type
  \item \verb$lookup_direct_dependency(fun_id)$: Finds all direct dependencies from a given function id
\end{itemize}

Using these data structures we can access a pointer with access to all lazy or direct dependencies using $O(1)$ expected time if we use a hash table using perfect hashing.~\cite{paper:perfect-hashing}. In worst case the space used is the maximum number of edges $O(|u| \cdot |o|)$, where $|u|$ denotes the number of notes representing underlying data and $|o|$ represents the number of nodes representing objects.

% subsection declaration-dependence-graph end

\subsection{Instance Dependence Graph}
\label{subsec:instance-dependence-graph}

The Instance Dependence Graph (IDG) is also an extension of the simple Object Dependence Graph (ODG), where the data entities corresponds to underlying data and the cached object instance generated from the cachable functions corresponds to the objects. An edge from underlying data $u$ to an object $o$ denoted ($u$, $o$) indicates that if $u$ is changed then $o$ must be updated.

The instance dependence graph for the running example is illustrated on figure~\ref{fig:instance-dependence-graph}. In the given example, this graph is quite primitive. It's worth noting that the edges in the IDG are instance specific representations of the direct dependencies of the DDG seen on figure~\ref{fig:declaration-dependence-graph}.

\begin{figure*}[ht!]
  \centering
  \includegraphics[width=0.3\linewidth]{figures/instance-dependence-graph.pdf}
  \caption{An example of an Instance Dependence Graph based on the running example}
  \label{fig:instance-dependence-graph}
\end{figure*}

The representation of the IDG is similar to the representation of lazy dependencies for the DDG, where the dependencies are represented using an outgoing adjacency list indexed by a hash table as illustrated on figure~\ref{fig:idg-data-structure}.

\begin{figure*}[ht!]
  \centering
  \includegraphics[width=0.6\linewidth]{figures/idg-data-structure.pdf}
  \caption{An illustration of the data structure representing the IDG on figure~\ref{fig:instance-dependence-graph}}
  \label{fig:idg-data-structure}
\end{figure*}

New dependencies are added using \verb$add_dependency$ described for the DDG in section~\ref{subsec:declaration-dependence-graph}, where the entity id is the source and the cached object is the target.

The queries needed for automatic invalidation are:

\begin{itemize}
  \item \verb$lookup_cached_object(entity_id)$: Finds all cached objects depending on the \verb$entity_id$.
\end{itemize}

% TODO: Analyze time and space (not the same as DDG since it depends on the data)

% subsection instance-dependence-graph end

% section dependency-management-for-cachable-functions end

\section{Dependency Registration}
\label{sec:dependency-registration}

For the application to know what and when to invalidate, the application must register the dependency declarations and name of cached functions instances. The dependency declarations that defines the lazy and direct dependencies are part of the source code and therefore available when the application is started as illustrated on figure~\ref{fig:declaration-dependency-registration}. This registration flow simply collects all the cached functions registered through the cachable function procedure and adds them as dependencies in the DDG.

\begin{figure*}[ht!]
  \centering
  \includegraphics[width=1.0\linewidth]{figures/declaration-dependency-registration.pdf}
  \caption{The flow in which lazy and direct dependencies are registered from the declarations}
  \label{fig:declaration-dependency-registration}
\end{figure*}

The registration of cached object instances happens when the cached object is accessed for the first time as illustrated on figure~\ref{fig:instance-dependency-registration}. In this flow we start by looking up the given cached object in the IDG, but since it's the first time the object is accessed it does not exist. To generate the name of the cached objects we use the ordered direct dependencies from the DDG combined with the input from the request. We derive dependencies to underlying data from the input that represents entities and add dependencies between the given entities and the cached object through the IDG.

\begin{figure*}[ht!]
  \centering
  \includegraphics[width=1.0\linewidth]{figures/instance-dependency-registration.pdf}
  \caption{The flow in which cached object instances are accessed when they are accessed the first time}
  \label{fig:instance-dependency-registration}
\end{figure*}

\section{Invalidation Propagation}
\label{sec:invalidation-propagation}

The purpose of invalidation is to be able to evaluate the freshness of a given cached object and know which cached objects are stale and need to be updated. A cached object is considered stale when it's underlying data has been updated, which happens during the request from a client that involves updating, inserting or deleting data in the primary storage. To be able to react to these events, Smache subscribes to callbacks from the database wrapper. When the database transaction succeeds the database wrapper will notify Smache with the id and type~\footnote{Type is another word for the relation in a relational database and a collection in a document-oriented database} of the manipulated entity. This notification will trigger cache invalidation through Smache.

Based on the type and id of the changed entity, Smache updates affected cached objects in two steps:

\begin{enumerate}
  \item Find keys for cached objects depending on the changed entity
  \item Invalidate both set of keys using timestamp invalidation
\end{enumerate}

There are two sets of object names found from the direct dependencies lazy and dependencies. To find the cached objects that directly depends on the given entity we query the IDG data structure with \verb$lookup_cached_object(entity_id)$, which returns a list of names for cached objects affected by the given change. The other set of object names are found by evaluating lazy dependencies for the given entity type.
The lazy dependency declarations are found by querying the DDG with \verb$lookup_lazy_dependency(entity_type)$. The lazy dependency is then evaluated using the declared procedure, which returns a new set of entities directly related to the given data source. We then query the IDG with \verb$lookup_cached_object(entity_id)$ with id's from the new set of relations, which returns lists of all cached objects that depends on the given entities. From these cached objects we filter all the keys such that we only include keys for cached objects related to the lazy dependency.
At last we invalidate both set of keys using timestamp invalidation as explained in the following section~\ref{subsec:timestamp-invalidation}.
Since the process of finding keys for depending objects involves multiple network calls to the primary storage, the invalidation propagation must be asynchronous to avoid a performance degrade for requests that involves updating underlying data.

% Analyze the queries performed in the algorithm as prequel to the next chapter

\subsection{Timestamp Invalidation}
\label{subsec:timestamp-invalidation}

To invalidate correctly the technique require the following liveness and safety property:

\begin{itemize}
  \item \textbf{Safety}: \emph{The value representing a fresh cached object must be based on the newest version of the underlying data}
  \item \textbf{Liveness}: \emph{A stale object must eventually be invalidated}
\end{itemize}

In trigger-based invalidation where the name of a cached object is considered static we need external information to evaluate whether a given cached object is fresh. In a naive invalidation technique, we can use a boolean value that indicates whether or not the cached value is fresh. When invalidation is triggered the value is set to \verb$false$ and when the cached object has been updated it would be set to \verb$true$. The problem here is that in an environment with multiple application servers, the technique would be prone to race conditions as illustrated on figure~\ref{fig:trigger-based-concurrency-problem}. Since the given cached object would be marked as fresh even though the value of the object in reality is stale, the cached object will not be invalidated until a new trigger is invoked and thereby contradict the liveness property.

One solution to this problem is to lock the invalidation mechanism such that the invalidation indication cannot be changed during an update. By using a lock the technique achieves liveness, but it also means that when another process wants to invalidate, it has to wait until the lock has been released. If the waiting process is a web server serving a user, the user would have to wait for the computation to finish.

To accommodate the liveness property and avoid user's having to wait for invalidation, we use a solution suggested in~\cite{paper:ibm-extended} that uses a form of logical timestamps to represent the version of a cached object. In this technique the cache system keeps track of the number of times a given cached object has been invalidated, which we denote $num\_of\_updates$. The number starts with $0$ and is incremented every time a cached object's underlying data changes.
We also keep a value representing the timestamp in which the current value of a cached object is based on, which we denote we $last\_update\_timestamp$. When the cached object is recomputed, the $num\_of\_updates$ is fetched and set as the timestamp of the current computation denoted $current\_computation\_timestamp$.
When the computation returns the new value for the cached object, we write update the cached object unless it has been updated during the computation. We can do this using the following algorithm:

\begin{enumerate}
  \item Fetch the value for $num\_of\_updates$ again denoted $latest\_num\_of\_updates$.
  \item
    \begin{enumerate}
      \item If $latest\_num\_of\_updates \geq current\_computation\_timestamp$ then we do nothing since a computation based on newer data has already updated the cached object.
      \item Else we update $last\_update\_timestamp$ to be equal to \\ $current\_computation\_timestamp$ and update the value.
    \end{enumerate}
\end{enumerate}

We can then say the following about the freshness of a cached object:

\begin{itemize}
  \item \textbf{A cached object is fresh} when\\$num\_of\_updates = last\_update\_timestamp$
\end{itemize}

To avoid race conditions the execution of this algorithm need to be executed in a transaction such that if $num\_of\_updates$ is incremented between step 1 and 2 then the algorithm be aborted. We assume that the execution of a computation at a given timestamp always yields the same result, which means we can retry the algorithm with the same result. In other words the algorithm is considered idempotent. We will therefore retry the update algorithm until it has finished. Alternatively we could lock all the values while executing the algorithm, but this means we will block the incremental of $num\_of\_updates$ that is responsible for invalidating and therefore more important.
Furthermore we also need the incremental of $num\_of\_updates$ to be atomic.

\subsubsection{Proof}
\label{subsubsec:proof}

Since the technique has not been proved for correctness in~\cite{paper:ibm-extended} and because the correctness of Smache relies on this technique, we will give a proof of correctness for timestamp invalidation by proving the liveness and safety requirements described above. In these proves we assume that the invalidation and update algorithms are executed atomically and that the invalidation happens immediately after underlying data has been changed.

\textbf{Safety}

We will prove \emph{safety} by contradiction by \emph{assuming we have a cached object considered fresh that holds a stale value}.

\begin{enumerate}
  \item We assume that $num\_of\_updates = t_i$ and since we know that a cached object is considered stale when $num\_of\_updates = last\_update\_timestamp$, we also have that $last\_update\_timestamp = t_i$
  \item For this to happen a computation $f_i$ starting when $num\_of\_updates$ was equal to $t_i$ ended up updating the cached value to be $v_i$ and $last\_update\_timestamp$ to be $t_i$.
  \item Since $v_i$ by definition is a fresh value, the value of the cached object can only become stale in on of the following cases:
    \begin{enumerate}
      \item[a)] Another computation $f_j$ updated the value of the cached object after the computation from step 2 updated the cached object
      \item[b)] The underlying data was updated after the computation from step 2 was started
    \end{enumerate}
  \item If the computation $f_j$ resulted in a stale value it must have been started at a point where $num\_of\_updates$ was equal to $t_j$, where $t_j < t_i$. This would contradict the algorithm since $f_j$ would not update the cached object with its stale value since $num\_of\_updates$ must have been equal to $t_i$ since it updated the cached object after $f_i$ and $t_i > t_j$.
  \item If the underlying data was updated after the computation was started when $num\_of\_updates$ was equal to $t_i$ then $num\_of\_updates$ must have been updated to be $t_k$, where $t_k > t_i$ after which we have a contradiction since $last\_update\_timestamp = t_i$ and the cached object is therefore not fresh since $num\_of\_updates \neq last\_update\_timestamp$ since $t_k \neq t_i$.
\end{enumerate}

\textbf{Liveness}

The \emph{liveness} property directly holds from the fact that $last\_update\_timestamp$ cannot be set to a value larger than $num\_of\_updates$ and since we assume that $num\_of\_updates$ is incremented immediately after an update to underlying data then we have that $num\_of\_updates > last\_update\_timestamp$ in the moment after the update. Since $num\_of\_updates \neq last\_update\_timestamp$ the cached object is stale and has therefore been invalidated.

% subsubsection proof end

% subsection timestamp-invalidation end

\subsection{Data Maintained by The Cache}
\label{subsec:data-maintained-by-the-cache}

To support timestamp invalidation we will extend the representation of a cached object with the following values:

\begin{itemize}
  \item $value$: The result of the cached function related to the cached object
  \item $num\_of\_updates$: The number of updates the given cached object is aware about
  \item $last\_update\_timestamp$: The timestamp corresponding to the computation of the current $value$
\end{itemize}

By representing the timestamps in the same object as the value we can more easily implement concurrency control mechanisms such as a lock or transactions as described above. We could represent the timestamps in another database, but this would mean we had to support distributed transactions across the cache database and the database with timestamps. We could also represent the timestamps in seperate objects than the value, but this would make it harder to shard the cached objects across multiple cache servers since we need to ensure that the cached objects are on the same server as the value to avoid distributed locking.

% subsection data-maintained-by-the-cache end

\subsection{Database Wrapper Triggers}
\label{subsec:database-wrapper-triggers}

% Trigger Invalidation Through Database Wrapper
% - Discuss the different alternatives and say why we want that

We've already discussed existing approaches in section~\ref{subsubsec:triggering-cache-invalidation} that compares the techniques on figure~\ref{fig:invalidation-trigger-comparison}. The triggers sent directly from the database or through a database sniffer have an advantage of capturing all changes made to the database - also those not sent through the application. The disadvantage of those techniques is that they are coupled to the database technology used. Instead of using a technique that depends on the exact database technology, Smache uses the database wrapper to trigger invalidations. The database wrapper makes it easier to change the implementation and thereby the database technology, which means the caching system becomes more flexible and easier to use in applications using different technologies. Furthermore since the triggers are intercepted through function callbacks in the web server process, the system doesn't need an external process to convert triggers from the database or sniffer into invalidations.

% subsection database-wrapper-triggers end

\section{Implementing Automatic Invalidation}
\label{sec:implementing-automatic-invalidation}

The Smache library implements automatic invalidation as described in the design above except for a few details such as the exact data structures of the dependency graphs, which have been replaced by similar representations to simplify the implementation. In this section how cachable functions (section~\ref{sec:implementing_cachable_functions_in_python}) are extended to support automatic invalidation. We will start by explaining how the IDG and DDG are implemented (section ~\ref{subsec:dependency-graphs}) followed by a description on how we implement transactions to update cached objects with timestamp invalidation (section~\ref{subsec:update-transactions}). Finally, we explain how we implement asynchronous invalidation to avoid invalidation affecting the performance of requests updating underlying data (section~\ref{subsec:asynchronous-invalidation}).

\subsection{Dependency Graphs}
\label{subsec:dependency-graphs}

The dependency graphs are used by the update procedure to derive relevant cache objects to be invalidated.
The DDG is static and will not change after the application starts and we can therefore easily duplicate the data across all the web server without implementing any replication since the data will not change. The DDG is therefore represented using objects in the application that can be queried by calling the methods of the DDG repository.
The IDG is dynamic in the lifetime of the application and will therefore be represented in a separate process. Optimally the IDG should be represented using hash tables, but to simplify the architecture, the data structure is implemented using the cache database that is assumed to be a common cache server supporting \verb$LOOKUP$ and \verb$STORE$ as described in section~\ref{subsec:cache-server}. In this implementation the keys corresponds to keys in the hash table and the linked lists are represented in the content by a serialized list.

Alternatively we could represent the dependency graphs using a cache manager process, but this would just introduce a potential bottleneck as well as a single point of failure. In our solution we limit the number of possible process failures by integrating the dependency graph into existing components of the architecture.

% subsection dependency-graphs end

\subsection{Update Transactions}
\label{subsec:update-transactions}

To implement timestamp invalidation we need to implement concurrency control to ensure that the timestamp invalidation is implemented correctly. In the version of Smache implemented during this thesis, we use \emph{Redis}~\footnote{http://redis.io/} as the cache database. Redis have support for simple transactions using a combination of the \verb$WATCH$-command and the \verb$MULTI$-command. The \verb$WATCH$-command is executed with a key, which tells Redis that a given transaction should fail if the specificed key is modified during the transaction. The \verb$MULTI$-command allows to specific multiple commands in a single atomic action.
Code snippet~\ref{code:timestamp_invalidation} is a simplified version of the code that implements the update procedure with timestamp invalidation. It uses the Python Redis~\cite{docs:python-redis} library to communicate with the Redis database. When it invokes the \verb$WATCH$-command the client will request Redis to start a transaction. Afterwards we build a \verb$MULTI$-command that will update the cached object if it is newer and invokes the \verb$EXECUTE$-command. The Redis-client will send the command and if the transaction fails it will raise a \verb$WatchError$-exception indicating the key has been modified, after which we retry the transaction with the same values.

\begin{figure*}[ht!]
  \input{code/timestamp_invalidation.py}
  \caption{The code for implementing timestamp invalidation in Python. RedisConnection represents an object communicating with the (Redis) cache database. CacheStore represents the object that communicates with the cache database.}
  \label{code:timestamp_invalidation}
\end{figure*}

% subsection update-transactions end

\subsection{Asynchronous Invalidation}
\label{subsec:asynchronous-invalidation}

Recall figure~\ref{fig:automatic-invalidation-flow} illustrating the flow where changes to the data in the primary storage system triggers a notification to the caching system, which invalidates affected cache objects. If the notification in step 3 was invoked synchronously then the performance of step 4 and 5 would affect the response time of the request and thereby affect the user experience of the use case involving the given request. In our case we cannot ensure to retrieve dependencies fast since we have lazy dependencies that must be evaluated based on requests to the primary storage. We will therefore implement the trigger-step asynchronously such that requests involving updates to underlying data are not affected by Smache.

If the notification is delivered more than once we will just achieve a worse cache hit rate and affect performance, but it will not affect the integrity of the cache object. We must therefore monitor and ensure exactly-once-delivery of the notification.
With most technologies the simplest solution for implementing asynchronous behaviour is to start a new thread that executes the invalidation. This way the asynchronous behaviour is controlled by the web server and we don't have to introduce new components to the architecture. Although this would be the preferred solution the Python language does not implement real concurrent behaviour, because it implements a Global Interpreter Lock that prevents multiple threads to execute the same bytecode at once.
To implement concurrent invalidations in Python we will therefore use the concept of \emph{Background jobs}, where the notifications are represented by a \emph{job} that is pushed into a queue. To execute the jobs we have multiple \emph{workers}, which are basically processes that also contains a version of the source code for the application. When there are jobs in the queue, a worker will pull the given job from the queue and execute it. Figure~\ref{fig:background-workers} illustrates how background workers are used to invalidate cache object asynchronously. To implement this behaviour we need to add application workers to the architecture as illustrated on figure~\ref{fig:architecture-with-workers}.

\begin{figure*}[ht!]
  \centering
  \includegraphics[width=1.0\linewidth]{figures/architecture-with-workers.pdf}
  \caption{The architecture required by a web application that uses Smache with automatic invalidation.}
  \label{fig:architecture-with-workers}
\end{figure*}

\begin{figure*}[ht!]
  \centering
  \includegraphics[width=1.0\linewidth]{figures/background-workers.pdf}
  \caption{How background workers are used do perform invalidation asynchronously.}
  \label{fig:background-workers}
\end{figure*}

% subsection asynchronous-invalidation end

% section implementing-automatic-invalidation end

\section{Summary}
\label{sec:invalidation-summary}

In this chapter we suggest an extension to cachable functions that provides automatic invalidation. The caching system uses the dependency information given as arguments, when a function is made cachable, to derive the dependency strategy for how and when to invalidate the related cached object. To provide efficient and fast invalidation, the caching system uses extensions of the Simple Object Dependence Graph data structure called Declaration Dependence Graph (DDG) and Instance Dependence Graph (IDG), to access the dependency information fast.
The caching system registers static dependency information stored in the DDG when the application starts and dynamic dependency information in the IDG when a cached function is called.
When underlying data are changed, the caching system will query the dependency data structures to find the cached object instances affected by the change, which are then invalidated using timestamp invalidation.

% section summary end

% chapter automatic_cache_invalidation end

