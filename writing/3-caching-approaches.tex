\chapter{Caching Approaches}
\label{chapter:caching}

Since caching is an approach widely used in practice there already exists multiple caching approaches described in literature, articles on the internet and in open source code. To discover existing solutions for the problems and challenges faced in the thesis, we will research and evaluate existing caching approaches. The approaches will be analyzed and evaluated based on the criteria described in section~\ref{sec:evaluating_caching_techniques} and  requirements from section~\ref{sec:requirements}.

As already the described the most difficult part of caching is the invalidation. We will therefore start by describing existing invalidation techniques followed by how the cached values are updated and related problems. We will also describe some specific caching approaches related to web development and finish the chapter by giving an approach to picking an appropriate caching approach for a use case based on the caching criteria.

\section{Invalidation Techniques}
\label{sec:invalidation_techniques}

When the underlying data for a cached value is updated we consider the cached value as invalid. Although this sounds as a trivial part, the mechanism for invalidating the cached value can become complex depending on the requirements for the use case. To be able to navigate the existing solutions we will cover the general invalidation techniques as well as the more advanced techniques described in literature.

% section invalidation_techniques end

\subsection{Expiration-based Invalidation}
\label{subsec:expiration_based_invalidation}

In cases where it is accepted to keep the cached values stale up to a given time interval, we can use the expiration-based invalidation technique, which is the simplest invalidation mechanism since the invalidation only depends on time and not updates of underlying data. Using the expiration-based invalidation we give up consistency and the level of freshness depends on the time interval set by the developer, but we are able to achieve immediate responses.

The expiration-based invalidation works by assigning a TTL (Time to Live) to the cached value. At some point when the TTL has expired, the cached value is invalidated. The responsibility of invalidation cached values with TTL is often placed on the cached database by piggybacking the TTL to the cached value when it is stored. This is for example supported by Redis and Memcached.

To give a sense of how the relation between when a cached value is considered valid (it is served from the cache) and when it is actually fresh, the timeline model has been applied to the expiration-based invalidation on figure~\ref{fig:timeline:expiration-based}.

\begin{figure*}[ht!]
  \centering
  \includegraphics[width=1.0\linewidth]{figures/timeline/expiration-based.pdf}
  \caption{The lifecycle of the expiration-based invalidation technique}
  \label{fig:timeline:expiration-based}
\end{figure*}

% subsection expiration_based_invalidation end

\subsection{Key-based invalidation}
\label{subsec:key_based_invalidation}

When the cached values are required to be up to date with the primary storage, we must ensure that when underlying data is updated, the depending cached values must be recomputed before the cached value can be served. The key-based invalidation gives these guarantees by giving up immediate responses since the users have to wait for computations to finish if they request an invalidated cached value. As mentioned in section~\ref{sec:evaluating_caching_techniques} the impact of this trade-off is proportional to the cache miss rate. This means that the key-based invalidation will not be suitable in cases where the computation time is too long or in cases where the cached values are updated too frequently.

Key-based invalidation works by constructing the cache key from parts of the underlying data such that the when the cached object should change, then the key also changes.~\cite{blog:key-based-invalidation}. The cached content is considered immutable and only have to be written once. This simplifies version management from the perspective of cache storage since there is no chance you read stale values if the key is assumed to be derived from the most recent version of underlying data.

The challenge of this method is to construct the key. To use this technique correctly (i.e. obtain the guarantees promised) the developer must construct a key that is ensured to change when the cached value is considered stale. Furthermore to obtain a maximum hit rate, the key must not change when the cached value is fresh. Given that the key is optimally constructed, the timeline looks as on figure~\ref{fig:timeline:key-based}.

\begin{figure*}[ht!]
  \centering
  \includegraphics[width=1.0\linewidth]{figures/timeline/key-based.pdf}
  \caption{The lifecycle of the key-based invalidation technique}
  \label{fig:timeline:key-based}
\end{figure*}

In the web application framework, Ruby on Rails, the key construction is simplified by using a key that includes the timestamps of the last update on some underlying data. An example of this technique can be seen in code example~\ref{code:key-based-invalidation}. In this example we use the type, id and update timestamp as parts of the key. This means that the cached value is considered invalid if it is called with an entity that has a different type, id or update timestamp. The intuition behind these components is that we want a unique cache value for each entity and we want the value to be recomputed when the entity is updated.

\begin{figure}
\input{code/key_based_invalidation.py}
\caption{Example of the key-based invalidation technique}
\label{code:key-based-invalidation}
\end{figure}

The key-based invalidation performs invalidation at the moment, the cached value is requested, but it is considered invalid at the moment after the cache key components are updated (e.g. the \emph{updated\_timestamp} in the code example). This means we get interval the value is considered value and fresh are always overlapping, which means we have consistency.

A caveat of this method method is that it generates cache garbage since old versions of a cached value are not removed. To avoid the complexity of keeping track of the relations between the different, the responsibility for cleaning up is moved to the cache database. Fortunately cache databases (such as Redis and Memcached) implements different such cleanup algorithms that detects obsolete values based on some policy. One such policy called \emph{Least Recently Used (LRU)} removes the cached values that are least recently used by keeping a timestamp of when the cached values where accessed last. Another policy called \emph{Least Recently Used (LRU)} keeps track of the frequency in which the different values are accessed and removes the ones that are least frequently accessed.

% TODO: Find litterature for these algorithms

% subsection key_based_invalidation end

\subsection{Trigger-based Invalidation}
\label{subsec:trigger_based_invalidation}

Instead of invalidating the cached value when requested, the cached values can be invalidated based on certain events triggered when the underlying data is updated. Given that invalidation triggers are located at all the places where the underlying data is updated, we can achieve the same guarantees of consistency and freshness as with key-based invalidation. The timeline model of the trigger-based invalidation on figure~\ref{fig:timeline:trigger-based} therefore looks similar to the key-based invalidation

\begin{figure*}[ht!]
  \centering
  \includegraphics[width=1.0\linewidth]{figures/timeline/trigger-based.pdf}
  \caption{The lifecycle of the expiration-based invalidation technique}
  \label{fig:timeline:trigger-based}
\end{figure*}

Where the key is used as an invalidation mechanism in key-based invalidation, the key is static for trigger-based invalidation i.e. the key for identifying a cached value does not change in its lifetime. The actual key becomes simpler since it is only responsible for uniquely identifying the cached value, but the localization is still complex since the key needs to be shared between the triggers and the places where the cached value is accessed. The responsibility of invalidating the keys are then moved to the definition of event triggers.

The simplest triggers are write-through updates, which are manual triggers inserted by the developer at places where the cached values should be invalidated. This require all developers to keep track of all places where underlying data changes, and furthermore be sure that the manual triggers are inserted when new code is introduced.

The simplest type of triggers are manually defined code that invalidates a given key. The code example seen on figure~\ref{code:manual-trigger-invalidation} illustrates how the manual triggers works. In practice the manual code triggers are often placed right after updates to the underlying data. Although this method is simple it often requires a lot of effort from the developer and is prone to errors since it requires global reasoning of the application to identify the places where underlying data is updated.

\begin{figure}
\input{code/manual_trigger_invalidation.py}
\caption{Example of how trigger based invalidation works with manual code invalidation}
\label{code:manual-trigger-invalidation}
\end{figure}

Due to the impracticality of managing the triggers manually a lot of research effort have been put into generating the triggers automatically, which will be the subject of the next section.

% TODO: Write this as if trigger-based invalidation is the overvall theme
% and then we have automatic invalidation that is a specialization of trigger-based invalidation
% where the triggers are created automatically based on some information

% subsection trigger_based_invalidation end

\subsection{Automatic Invalidation}
\label{subsec:automatic_invalidation}

% BELOW: OLD

A lot of work has been put into using triggers from the database to invalidate cached data. \cite{paper:cache-genie} suggests a solution based on the Object Relational Mapper programming technique to capture relevant triggers. \cite{paper:deploy-time} also suggests using a database wrapper in the application-layer that captures and analyzes database commands and use them as triggers. TxCache suggested in~\cite{paper:liskov} uses daemon processes to monitor the database for relevant triggers. This method has the advantage that it allows multiple types of applications to manipulate the same database as opposed to~\cite{paper:cache-genie, paper:deploy-time}, where the triggers would not have been captured if the database command was made around the application. On the other hand it introduces complexity of running and monitoring the processes. With relation to database monitoring this introduces a trade-off between the complexity of the system and an assumption that multiple types of applications does not alter the same data.

% TODO: Consider if relevant
% In some architectures the changes to the system are based on triggers or events. One such architecture called Event Sourcing works by using Domain Events to describe the changes of the system instead of using database commands~\cite{blog:focusing-on-events}. Since these events are a natural part of the application, they can be used as invalidation triggers without further additions.

In~\cite{paper:liskov} Dan Ports et. al. uses database triggers to achieve transactional consistency for application-level caching, which ensures that any data seen within a transaction, shows a slightly stale but consistent snapshot across the storage and cache system. The database triggers are implemented using two database daemons that monitors a slightly modified version of PostgreSQL. The suggested solution, called TxCache, promises a very strong consistency guarantee, but it also comes with assumptions about the storage system and cache system used, and requires additional running daemons, which makes the full system more complex to run reliably. Furthermore these requirements contradicts flexibility and adaptability since the system assumes specific properties from the storage system and cache system, which makes it more difficult to change these components and adopt the caching system if an existing system does not use the given components.

Another solution proposed by Chris Wasik et. al.~\cite{paper:deploy-time} uses deploy-time analysis of the code to detect dependencies between the cached functions and the dependent relations. To invalidate the cached functions automatically, the system injects code that invokes relevant invalidation callbacks in places where the underlying data is updated. Where~\cite{paper:liskov} suggests a system that comes with requirements for the architecture and technologies used, the deploy-time model is a simple system that is able to use simple key-value stores for caching and any SQL storage system. But as oppose to~\cite{paper:liskov} it does not result in as strict consistency guarantees. But despite of being a simple method, the deploy-time model is based on a system where the source code changes for different environments, which could cause errors in one environment and not in others. As an example, the code in a development or test environment could work as expected but still result in errors when deployed to a production environment, where the code is injected. Even though the deploy-time solution avoids single points of failures as with a cache manager, it needs additional operations that have to be executed in the existing procedures. In a system with complex dependencies between the procedures and underlying data, the generated source code could decrease performance that cannot be optimized by the developer using the system.

CacheGenie is another cache system described by Priya Gupta et. al. that uses the Object Relational Mapping (ORM) library to detect changes made to the database. Some ORM libraries already implements these triggers, which makes this approach easier to integrate into web applications that uses ORM libraries, since the caching library does not rely on database monitors. CacheGenie tries to solve the problem of managing cache invalidation when caching database queries, by letting the developer predefine cached queries that are automatically updated in the application. CacheGenie is also based on a simple model, but each the cache definitions are based on assumptions about the specific queries and cannot be used to cache objects of a more coarse granularity.

On the other end of the granularity scale,~\cite{paper:db-driven-http} suggests a system that caches HTTP responses. It uses a sniffer process that monitors the lifetime of a HTTP request with the queries made to the storage system. Through the information captured by the sniffer, the system builds a table that maps a given HTTP resource to the queries made. The system then caches the HTTP resource that is invalidated when underlying data related to the given resource changes. This method is interesting since it allows to cache without changing the code of the web application, but it is only described at the granularity of HTTP responses since it uses the communication between the web application, storage system and cache to achieve automatic invalidation.

Jim Challenger et. al. has written multiple papers on the system used for the content management website in the Olympic Games in 1998 and 2000~\cite{paper:ibm, paper:ibm-extended}. The system is based on content that are all precomputed when served to the user, which resulted in a system that scaled for many users with content served fast since the web server only had to find the appropriate cached article when serving content. In order to allow editors to change articles and fragments, the system introduces the Data Update Propagation (DUP) algorithm. DUP uses an Object Dependence Graph (ODG) that describes the relationship between fragments using a Directed Acyclic Graph (DAG). The ODG describes both the relationships of how the fragments are embedded in each other and relationships describing the hypertext links between articles. To avoid race conditions and hypertext links to missing fragments, the fragments need to be updated in a specific order. More specific when a fragment f1 that embeds another fragment f2, the system need to update f2 before f1. Since the ODG is described DAG there is always a topological order of the nodes, which satisfies the described property for any node. The system runs using a CMS system, where the content is defined using a CMS system and not using functions from the source code. This simplifies the challenge of persisting the cached content since it does not change when a new version of the source code is deployed. It therefore leaves the challenge of updating cached content, when the definition of the computations changes.

% TODO: Make below content fit in section

A more flexible technique is to cache functions. \cite{paper:liskov} describes a programming model for cacheable functions that essentially is functions annotated as cacheable. Although this seems attractive, it has limitations with respect to the procedures executed in the function. \cite{paper:liskov} describes the requirement of cachable functions of their programming model: “To be suitable for caching, functions must be pure, i.e. they must be deterministic, not have side effects, and depend only on their arguments and the storage layer state.” By this definition they explain that the storage layer state are treated as implicit arguments and thereby reach the classical definition of a pure function that is a transformation that always gives the same output from the same input.

\cite{paper:deploy-time} suggests a similar programming model, where the relationships of the underlying data has to be explicitly annotated, but where the rest of the caching system is much simpler than the one described in~\cite{paper:liskov}.

The requirement for strong consistency introduces complexity as seen with the trigger-based and key-based cache invalidation. Some objects can be cached with weak consistency, which allows much simpler caching techniques. One method is to assign a TTL (Time to Live) to the cached value. At some point when the TTL has expired, the cached object is invalidated. The invalidation can be enforced by the cache database (Redis and Memcached supports this - TODO: include references) or as part of the protocol between the client and server as with HTTP-caching explained in section~\cite{paper:web-caching-schemes}.

% subsection automatic_invalidation end

\subsection{Write-Through Invalidation}
\label{subsec:write_through_invalidation}

% TODO: Write about: Fault-Tolerance since: what if the computation suddenly fails: then we can serve the old value.

% subsection write_through_invalidation end

\subsection{Hybrid Approaches}
\label{subsec:hybrid_approaches}

% TODO: Write something about combining different caching techniques to achieve satisfactory requirements

% subsection hybrid_approaches end

\section{Updating the Cache}
\label{sec:updating_the_cache}

\subsection{Data Update Propagation}
\label{subsec:data_update_propagation}

% TODO: Detailed explanation of the IBM update algorithm
% TODO: Consider if this should be postponed to the Data Update Propagation Chapter

% subsection data_update_propagation end

\subsection{Consistent Concurrent Write-Through}
\label{subsec:consistent_concurrent_write_through}

% TODO: Write this as a problem that is ignored in many existing caching solutions or maybe just traded off for simplicity. Describe how key-based invalidation ensures this and how all other could have this problem solved.

At first we would like the caching to be correct. In the case of caching, we will define correctness in terms of liveness and safety: the caching system will update the cache when necessary and it will eventually return the most fresh value computed. That is if we have a computation $f$ that computes the value $v_1$ at time $t_1$ and $v_2$ at time $t_2$ then the cache store will eventually contain $v_2$ given that $t_2 > t_1$. Although this could be seen as a prerequisite for the caching system, most implementations ignore this fact to achieve a simpler cache system. Because when we want to keep the integrity between updates we need some kind of ordering for the updates that adds complexity to the system. The problem is also illustrated on figure~\ref{fig:incorrect-updates-analysis}.

\begin{figure*}[ht!]
  \centering
  \includegraphics[width=1.0\linewidth]{figures/incorrect-update-analysis.pdf}
  \caption{Showing how two concurrent caching updates from two different application servers results in an inconsistent state. We see that even though the request from \emph{Web 2} are based on data older than \emph{Web 1} it gets to write }
  \label{fig:incorrect-updates-analysis}
\end{figure*}

% TODO: Analysis and argue why some of the solutions leads to inconsistent cached values and breaks the safety requirement.

% subsection consistent_concurrent_write_through end

% section updating_the_cache end

\section{Caching Approaches in Web Development}
\label{sec:caching_approaches_in_web_development}

% TODO: Rewrite this intro

One important aspect of caching is the granularity of the cached content. There is no doubt that it is most desirable to be able to cache any granularity, but since the cached content could be anything, the system cannot make any assumptions about cache management to e.g. allow for smarter invalidation. Therefore most existing work are based on a specific caching granularity from the data queried from the database to the HTTP response sent to the client.

\subsection{HTTP Caching}
\label{subsec:http_caching}

On the other end of the granularity scale, the developer could choose to cache the entire HTTP response send to the user. This could be the HTML documents served to the user as the website, but it could also be the JSON or XML response from an API. The HTTP protocol is the standard among web browsers to display web content and it’s widely used to communicate between web services. Since the HTTP protocol also include caching methods, which will be explained in the HTTP section, it is a very attractive caching technique among web applications.

The HTTP protocol includes multiple mechanisms for controlling cache consistency that allows the web server to implement both key-based and expiration-based cache invalidation. These mechanisms are controlled using HTTP headers. The expiration-based cache invalidation information about the cache date and age is specified using the Cache-Control header. The client is then able to derive if a given resource is valid at a given time or if the resource has to be refreshed. To use key-based invalidation, the web server can attach a tag that uniquely identifies a given version of a resources (e.g. using a hash of the content). When the client sends a new request, it attaches the ETag of the last version received, and the web server can now respond with a 304 Not Modified with no content. This tells the client it can safely use the last version.

Caching HTTP responses is a great technique when the same response are served to the multiple clients, but in situations where the content is updated often or personalized to each user, it becomes a less efficient technique since large documents are recomputed often. In the case where a small fragment of the content is personalized, it would be more desirable to only update that given fragment instead of recomputing the full document.

% subsection http_caching end

\subsection{Database Query Caching}

The database can be a bottleneck in the goal of achieving fast rendering of dynamic pages, because it’s often a requirement to have structured data at which you perform complex queries. In both cases the queries can become slow when the application need to scale with relation to data or users. Even though most storage systems allows indexing to optimize specific queries, it can still be difficult for the storage system to optimize in a space efficient way. One solution to this problem is to use query caches.

In~\cite{paper:transparent-caching, paper:cosar} this problem is solved using a proxy caching server between the web application and the database. This allows for transparent caching that require almost no changes to the system, but unfortunately it requires a lot of work to maintain the index used for cache validation and parse the queries received from the web application. Furthermore this solution are mostly made for relational databases with SQL language and require a new implementation for it to work on different storage technologies such as document-oriented databases.

\cite{paper:cache-genie} also describes a query caching solution, where the cache management is placed entirely in the application-layer. It is based on a common technology used in web application called Object Relational Mapper (ORM), where the data model is mapped to objects in the application and often the queries are made using methods on the object. Using the ORM in the Python framework Django, \cite{paper:cache-genie} implements an extension that allows common database queries to be cached and automatically updated. Compared to having a middleware caching layer, this solution has the advantage of being able to integrate with both different database technologies (within the capabilities of the ORM) and caching systems. On the other hand, it does not capture database updates made without using the ORM. This means if updates are made manually or another application uses the same database, the cached queries are not updated.

% TODO: Merge materialized views into this subsection

\subsubsection{Materialized Views}
\label{ssub:Materialized Views}

Where the query caches described so far are either middle-tier or on the application-layer, caching using materialized views occurs on the database layer. Materialized views are “virtual tables” generated from other data in the database. It works by storing queries explicitly declared by the developer. The virtual tables can be explicitly refreshed or update when the dependent data changes. Materialized views is a good solution for optimizing database queries, but since the computation occurs on the database level, the computation capabilities are limited by the database technology.

Caching database queries and materialized views allows for easy and transparent caching, but it does not allow computations on the web application, which limits the applicability.

% section caching_approaches_in_web_development end

\section{Choosing the right Caching Technique}
\label{sec:choosing_the_right_caching_technique}

% TODO: Explain/prove why this is a trade-off (ensure it is not written in caching model as well) - it probably is


% TODO: Remember both criteria and requirements

% Write a lot about the trade-offs:

The system need to cache the result of computations, which means it has to cache objects with a granularity more coarse than database queries. The HTTP protocol includes multiple features for cache management between the client and server, but it also makes the caching inflexible with relation to efficiency. In some situations HTTP responses includes shared fragments that need to be computed for each HTTP endpoint. Since the system expects long running computations, it would be more efficient to cache the result of those computations, meaning the system need to work on a granularity of fragments or functions. Since functions returns an output that could be considered a fragment, we will consider them as the same granularity.

In order to keep the cached values up to date we need automatic cache invalidation. In order to achieve automatic cache invalidation with transactional consistency,~\cite{paper:liskov} describes a solution a solution that need assumptions about the storage and cache system. The system designed in this thesis does not need such strict consistency guarantees, and some of the components in the solution is therefore not necessary.

The solution suggested by Jim Challenger et. al. is highly relevant to the problem of this thesis, but since the system is designed for a publishing system, it leaves some challenges that have to be solved to satisfy the requirements of this thesis.

Also relevant is the paper by~\cite{paper:deploy-time} that suggests a solution that identifies dependencies and injects invalidation callbacks into the source code on deployment. This method makes the caching transparent, but it also comes with the cost that the code will be different in development and production. Furthermore the solution described does not perform write-through updates, which would also be inefficient since it would slow down existing operations that involved cache invalidations.

Conclusion from this chapter: solutions exists for a lot of the sub-problems this thesis is facing, but there exists no complete solution to satisfy the requirements.

% section choosing_the_right_caching_technique end
