\chapter{Smache: Cachable Functions}
\label{chapter:smache-cachable-functions}


% Structure:
% - Existing approaches does not solve our problem optimally
% - Smache: Programming Model for Cachable Functions

\section{Existing Approaches Does Not Solve Our Problem Optimally}
\label{sec:existing_approaches_does_not_solve_our_problem_optimally}

% TODO: Find shorter title for this section

%   - State what we would like
%   - Not solved by the Deploy-Time method:
%     - Lack of fault-tolerance
%   - Dan Ports neither:
%     - Made for PostgreSQL (makes it difficoult to use other databases)
%   - IBM:
%     - Supports write-through, BUT it is made for published content, not functions
%   - Conclusion:
%     - We need a system that lies THERE!:
%       - Medium adaptable (= easy to adapt to new technologies)
%       - Supports Write-Through
%       - Flexible:
%         - Default: immediate response + relaxed freshness
%         - Other:
%           - Strict freshness + automatic invalidation + update in background
%           - Future:
%             - Expiration-based caching

% section existing_approaches_does_not_solve_our_problem_optimally end

\section{The Cachable Function Model}
\label{sec:the_cachable_function_model}

% Introduce the caching function and explain why it's THE solution
%   - Show a diagram with the architecture of the solution
%     => Two diagrams:
%       - request cycle
%       - invalidation + update cycle

\begin{figure*}[ht!]
  \centering
  \includegraphics[width=1.0\linewidth]{figures/cachable-function-control-flow.pdf}
  \caption{The control flow during a call to a function cached through Smache}
  \label{fig:cachable-function-control-flow}
\end{figure*}


%   - Application-layer definition
%   - Automatic Cache Invalidation (with listener)
%   - Data Update Propagation (with worker)

% - Definition of cachable function (taken from Dan Ports)

\subsection{Restricted to Pure Functions}
\label{subsec:restricted_to_pure_functions}

% TODO: Write about how the solution is restricted to pure functions,
%   - copy-paste pretty much (AND REFERENCE Dan Ports)

% subsection restricted_to_pure_functions end

\subsection{Making Functions Cachable}
\label{subsec:making_functions_cachable}

% TODO: Write about how the functions are converted to be cachable
%   - copy-paste pretty much (AND REFERENCE Dan Ports)

% subsection making_functions_cachable end

\subsection{Automatic Cache Invalidation}
\label{subsec:automatic_cache_invalidation}

% subsection automatic_cache_invalidation end

\subsection{Data Update Propagation}
\label{subsec:model_data_update_propagation}

% subsection data_update_propagation end

\section{Implementing Cachable Functions in Python}
\label{sec:implementing_cachable_functions_in_python}

\subsection{Defining the Cachable Functions}
\label{subsec:defining_the_cachable_functions}

% subsection defining_the_cachable_functions end

% section implementing_cachable_functions_in_python end

%   - Discussion about guarantees (where does it lie in the comparison mode)
%   - MAYBE wait until final discussion
%     - Configurations:
%       - Always Immediate Response + Relaxed Freshness
%       - Strict Freshness
%   - Python Implementation:
%     - Introduce decorators
%     - Show me the code - specifically use the running example!

% section the_cachable_function_model end

% chapter smache-cachable-functions end
