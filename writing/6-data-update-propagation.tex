\chapter{Data Update Propagation}
\label{chapter:data-update-propagation}

Until now we've explained how to build a caching system that is able to cache the result of functions, where the result is invalidated automatically when its underlying data changes. This solution makes it easier to manage cache invalidation and locate cached values, but after a result has been invalidated the user has to wait for the value to be recomputed, which can be critical for the user experience in cases where the computation time is too long.
In this chapter we will extend the current solution with in-place updates using a data update propagation (DUP) algorithm that schedules updates for invalidated cache objects addressing the second challenge of the problem description~\ref{sec:problem}. We will start by covering existing approaches (section~\ref{sec:existing-data-update-propagation-approaches}) followed by an analysis of the problems involved with concurrent in-place updates (section~\ref{sec:race-condition-on-write-through-invalidation}). Based on the knowledge from these sections we will describe the design (section~\ref{sec:the-data-update-propagation-algorithm}) and implementation (section~\ref{sec:implementing-the-data-updata-propagation-algorithm}) used in Smache.

\section{Existing Data Update Propagation Approaches}
\label{sec:existing-data-update-propagation-approaches}

In the approach suggested by Jim Challenger et.al.~\cite{paper:ibm, paper:ibm-extended, paper:ibm-publishing-system}, the DUP algorithm is based on invalidation from the Object Dependence Graph described in section~\ref{sec:simple-object-dependence-graph}. When underlying data changes all depending cached objects are scheduled for update, but since some of the cached objects depends on other cached objects they cannot be computed in any order. If a cached object $o_2$ that depends on another cached object $o_1$ where updated first then it would be based on an old version of $o_1$, which means it would still be stale and the algorithm would not have liveness. To solve this problem the approach updates all cached objects in a topological order, which ensures that $o_2$ is ordered after $o_1$ since it depends on $o_1$. One limitations of this technique is that it only works if the object dependence graph has no cycles i.e. it is an \emph{Directed Asyclic Graph}. Another limitation is that when there is an order of the jobs then they must be synchronized when executed in parallel.

TODOS:

- Write about Labrinidis's strategy \\
- Write about DBProxy

% A. Labrinidis and N. Roussopoulos~\cite{paper:update-propagation-strategies}
% Look for similar solutions in research!
% => "DBProxy: A dynamic data cache for Web applications"
% => "Update Propagation Strategies for improving the quality of the web"
% => ?

% section existing-data-update-propagation-approaches end

\section{Race Condition on Write-Through Invalidation}
\label{sec:race-condition-on-write-through-invalidation}

When the name of the cached objects are the same as in trigger-based invalidation, the cached object becomes a shared resources that have to be accessed and updated from multiple web servers. This means we have to consider the concurrency challenges related to this concurrent environment to avoid avoid race conditions( while ensuring liveness of the system).
As illustrated on figure~\ref{fig:incorrect-updates-analysis} a race condition can occur when there are processes updating the same cached objects at the same time. In this case a race condition affects the correctness of the system such that a given cached object is marked as fresh even though it's value is stale.

\begin{figure*}[ht!]
  \centering
  \includegraphics[width=1.0\linewidth]{figures/incorrect-update-analysis.pdf}
  \caption{Showing how two concurrent caching updates from two different application servers results in an inconsistent state. We see that even though the request from \emph{Update Process 2} are based on data older than \emph{Update Process 1} it gets to write }
  \label{fig:incorrect-updates-analysis}
\end{figure*}

% section race-condition-on-write-through-invalidation end

\section{The Data Update Propagation Algorithm}
\label{sec:the-data-update-propagation-algorithm}

% TODO: Consider this a chapter for itself
The existing solution described in section~\ref{sec:existing-data-update-propagation-approaches} assumes that a given cached computation is only accessed by one process at the same time. This has the advantage that they avoid race conditions, but it also means that the scheduling have to synchronize parallel execution to avoid updating the same object. The optimizations are then achieved using scheduling algorithms that evaluates the update jobs and prioritize them based on some metric such as the freshness or computation time. By having this assumption and enforce an order of execution some jobs have to wait to be executed. The only way to optimize the execution of these jobs is to acquire faster CPU's, which is expensive compared to buying more CPU's.

Instead of using advanced scheduling algorithms to achieve a high throughput we suggest a solution that allows concurrent updates of the same cached object. By allowing this we can scale our update execution by buying more CPU's and thereby be able to execute more jobs at the same time.

To allow concurrent updates we use Timestamp Invalidation as already explained in section~\ref{subsec:timestamp-invalidation}. Timestamp Invalidation ensures that a given computation does not overwrite the value of a cached object if the computation is based on a newer version of underlying data. Figure~\ref{fig:incorrect-update-analysis-timestamp-fix}

\begin{figure*}[ht!]
  \centering
  \includegraphics[width=1.0\linewidth]{figures/incorrect-update-analysis-timestamp-fix.pdf}
  \caption{How Invalidation Timestamps fixes the concurrency problem described in figure~\ref{fig:incorrect-updates-analysis}.}
  \label{fig:incorrect-update-analysis-timestamp-fix}
\end{figure*}

Since invalidation timestamps ensures the integrity of the cached objects we are allowed to scale horizontally and thereby execute as many jobs at once as there are update processes available. Furthermore since timestamp invalidation makes it possible to evaluate if a cached object is fresh, we can avoid computation of duplicates by only execute computations for stale cached objects. We could also optimize it even further by avoid scheduling recomputations for cached objects that are already scheduled, but this would require an atomic procedure to check if there already exists a similar job and only schedule the new job if there are none.


% Our approach:
% - Use invalidation timestamps (as described before!)
% - We can then paralellize

% Advantages:
% - We remove this assumption by using timestamp invalidation and always
%   recompute values for sub-computations when they are not fresh.
% - That is: Ensure that a given computation executed at time $t1$ always compute values
%   based on the state of underlying data at time $t1$.
% - This way we still have the possibility of optimizing the write-through by
%   scheduling, but we also have a simple scheduling algorithm that is easy to
%   parallelize and require no advanced algorithm.

%   - Avoid unnecessary updates
%     => Prune duplicate jobs? (spoiler: NOT)
%     => Do not compute fresh values (YES!)
%   - IBM:
%     => Schedule in topological order + parallelize sub-graphs
%   - Others:
%     => "Update Propagation Strategies for Improving the Quality of Data on the Web "
%   - Our solution:
%     => Using timestamp invalidation for concurrency control also for in-place updates
%     => Always compute newest value (do not use stale value of sub-computes)
%     => We get: retries, parallel computations, simple scheduling
%   - Maybe measure QoD?

% section the-data-update-propagation-algorithm end

\section{Implementing the Data Updata Propagation Algorithm}
\label{sec:implementing-the-data-updata-propagation-algorithm}

TODO: Write about how the data update propagation is implemented

\begin{itemize}
  \item Using queues (i.e. the producer/consumer model) as the scheduling mechanism (add job to queue to schedule update). Workers are then polling from the queue and executing.
  \item How the timestamp mechanism works (if it has not already been shown in the chapter on automatic invalidation)
  \item The architecture of the solution
\end{itemize}

% section implementing-the-data-updata-propagation-algorithm end

\section{Discussion}
\label{sec:discussion}

% section discussion end


% section updating_the_cache end
