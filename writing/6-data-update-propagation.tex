\chapter{Data Update Propagation}
\label{chapter:update_propagation}

\section{Updating the Cache}
\label{sec:updating_the_cache}


\subsection{Data Update Propagation}
\label{subsec:data_update_propagation}

% TODO: Detailed explanation of the IBM update algorithm
% TODO: Consider if this should be postponed to the Data Update Propagation Chapter

% subsection data_update_propagation end

\subsection{Consistent Concurrent Write-Through}
\label{subsec:consistent_concurrent_write_through}

% TODO: Write this as a problem that is ignored in many existing caching solutions or maybe just traded off for simplicity. Describe how key-based invalidation ensures this and how all other could have this problem solved.

At first we would like the caching to be correct. In the case of caching, we will define correctness in terms of liveness and safety: the caching system will update the cache when necessary and it will eventually return the most fresh value computed. That is if we have a computation $f$ that computes the value $v_1$ at time $t_1$ and $v_2$ at time $t_2$ then the cache store will eventually contain $v_2$ given that $t_2 > t_1$. Although this could be seen as a prerequisite for the caching system, most implementations ignore this fact to achieve a simpler cache system. Because when we want to keep the integrity between updates we need some kind of ordering for the updates that adds complexity to the system. The problem is also illustrated on figure~\ref{fig:incorrect-updates-analysis}.

\begin{figure*}[ht!]
  \centering
  \includegraphics[width=1.0\linewidth]{figures/incorrect-update-analysis.pdf}
  \caption{Showing how two concurrent caching updates from two different application servers results in an inconsistent state. We see that even though the request from \emph{Web 2} are based on data older than \emph{Web 1} it gets to write }
  \label{fig:incorrect-updates-analysis}
\end{figure*}

% TODO: Analysis and argue why some of the solutions leads to inconsistent cached values and breaks the safety requirement.

% subsection consistent_concurrent_write_through end

% section updating_the_cache end


% ## Consistency/Integrity
% BOTTOM: NOT TRUE FOR KEY-BASED
% WE CALCULATE AND STORE THE VALUE FOR THE GIVEN KEY, WE DO NOT HAVE STATE
% ABOUT FRESHNESS
% - One problem that neither automatic cache invalidation nor current
%   pull based cache invalidation takes care of is the potential race conditions
%   on data update. Consider e.g. the following events:
%
% - Request made
% - Cache not fresh
% - Start calculating new cache value (f1)
% - f1: Fetch newest data from primary db
% - Data updated => cache invalidated
% - f1: write cached value
%
% In this case, the data update invalidates the cached value based on data
% that is a later version than the data f1 used to update the cache. This would
% result in the cache having a value based on old data => Inconsistent
%


% ### Updating

Since we need to be able to handle multiple web servers, we need to consider
the system as a concurrent system.

The problem is basically that the key of the cached content is the same, which
means that we do not have any accounting of which update are allowed to
overwrite others. E.g. if two updates u1 and u2 happens right after each other
and executes a write through update with relevant data, where u2 will produce
a fresh result and u1 will not. If u2 is fast at computing and u1 is slow then
u2 will start by writing the fresh cache value, and u1 will overwrite the
fresh one with a stale result.

In this problem we need a mechanism such that u1 does not overwrite u2.

% #### Using lock

One solution is to use locks such that only one update can be computed per
key in a given moment. In the given case u2 would be locked until u1 has
finished and written to the cache.

The downside of this method is that the result can in staleness of time t(u1) + t(u2) - 1.

% #### Using timestamps

Another solution is to use timestamps. When u1 is scheduled, it fetches a
timestamp from resource (redis incremental). This timestamp is then submitted
when the cached value is written. In the given example u1 would get one
timestamp $t$ and u2 will then get $t + 1$. u2 will then submit a cached value
with timestamp $t + 1$ and afterwards u1 will try to submit a stale cached value
with timestamp $t$, but since a newer version has been written it would be ignored.

NOTE: This method is also mentioned in the IBM paper as amount of times a given
value has been written.

The downside of this method is that we need some kind of centralized timestamp
generator, which will then become a single point of failure of the system.
Theoretically this could be solved in a proper manner using a distributed
and replicated timestamp generator (ETCD maybe).

We need to assign timestamp in the execution. If we assigned timestamps before
we inserted into the queue and we retried a job, then it would be fair to
assume that the retried job had the newest data and therefore has to be able
to write to the cache.


% #### Better timestamp solution

We want the properties:

* Liveness: a given function is executed after finite time. In this case it
  means that if we have any set of updates $U$ that results in the global state
  $s$ and we have a cached value $C(f, s)$ that is the cached result
  of function $f$ for state $s$, then we know that at some point in time after
  the events in $U$ have occured, we must have stored $C(f, s)$ or $C(f, s')$
  where $s \rightarrow s'$.

If we used a naive algorithm, where a given update invokes an update execution
that then computes the value and updates the cache, we would have potential
race conditions and not ensure liveness. This could happen in the simple case

u => f

with event order:

* u1: u is updated => global state s
* f is invoked for worker w1 with state s
* u2: u is updated => global state s'
* f is invoked for worker w2 with state s'
* worker w2 finishes and updates f with the cached value C(f, s')
* worker w1 finishes and updates f with the cached value C(f, s)

In this case, the resulting value would be C(f, s) which would contract
liveness since we have a set of updates $U = {u1, u2}$ that gives the global
state $s'$, where the cached value $C(f, s')$ is never stored.

To solve this we can use timestamps:

Instead of incrementing timestamp as above, we use the notion of state timestamp
and value timestamp. There is one state and value timestamp for each cached
value. These are defined as:

* state timestamp: the amount of times the underlying data/state has been
  updated. When some underlying data is updated, this number is updated for
  all affected computed nodes.
* value timestamp: the state timestamp which the last update was based on.

The system should then do as following:

* R1 (value update): Right before the state input is fetched $cst_i$ and the function is executed,
  the state timestamp for the cached value is taken an called $ft_i = cst_i$. When the function
  has finished and wants to store the cached value, it need to ensure that
  the current value timestamp for the cached value $cvt_i$ holds: $cvt_i < ft_i$.
  If this holds, the cached value is written and $cvt_i := ft_i$. This is done
  atomically.
* R2 (freshness): When a cached value is written, We also set the cached value as fresh
  ($cf_i = True$) iff the current $cst_i = cvt_i$, else we set $cf_i = False$.


% ### Propagation

When one value is updated, we need to propagate the update. Since we do not
want to calculate an entry more than once, we need to do it in topological
order.

Another thing is:
- What if a function instance is set to be executed a lot of times?
  - Here we would like the topmost instance to be executed, because it
    means that it's siblings will be cached. Alternatively if only the
    topmost was executed, we could risk that some siblings are not updated,
    if e.g. there is a condition which removes the relation. Phew...
  - We also don't want to risk starvation, ie we want to ensure the instance
    is executed at some (fair) point in the future. If we e.g. removed every
    instance except the topmost a lot of times, it would result in starvation.

% #### Truncating the queue periodically

One solution would be to truncate the queue when a new job comes in (or periodically).

This would need some kind of algorithm to control and remove duplicate jobs:
One way would be to remove every duplicate job except for the first and last in line.
This would avoid starvation since the first job in line will never be removed, and
it will ensure that we eventually will calculate the newest entry. This assumes
that the queue will be finished in a fair amount of time. For example if 100
jobs with including the same function would be enqueued then it would mean we probably
mean that the truncated job would have a 97 jobs in between.

% #### Truncating the queue when inserting



% #### Just don't execute jobs, where the value is already fresh

% % chapter data_update_propagation end
