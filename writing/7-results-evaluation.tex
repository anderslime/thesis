\chapter{Tests and Evaluation}
\label{chapter:tests-and-evaluation}

\section{Experimental Setup}
\label{sec:experimental-setup}

% section experimental-setup end

\section{Cache Sizes and Performance}
\label{sec:cache-sizes-and-performance}

Possible tests (choose the ones that makes sense):

\subsection{Performance}
\label{subsec:performance}

\begin{itemize}
  \item Prove that update requests are not made slower (because they are async)
  \item Test performance when a given computation uses a lot of cached results (can we do multi-fetch here?)
  \item Test how long it takes to warm up cache
\end{itemize}

% subsection performance end

\subsection{Space}
\label{subsec:space}

\begin{itemize}
  \item Test memory consumption of cache database when using Peergrade.io
\end{itemize}

% subsection space end

\section{Update Throughput}
\label{sec:update-throughput}

\begin{itemize}
  \item Test and compare sequential topologically sorted with parallelized
  \item Test update throughput under different write-patterns
  \item Test how often cached functions are executed unnecessarily
  \item Test some QoD measurements
\end{itemize}

% section update-throughput end

% section cache-sizes-and-performance end

\section{Evaluation}
\label{sec:evaluation}

Discuss this:

\begin{itemize}
  \item Test results
  \item Software Design
  \item Adaptability
  \item (Fault-Tolerance)
\end{itemize}

% Fault-Tolerance:
%  - If the cache is not reachable:
%    - If computation time > 10s: should return no result
%    - If computation time < 10s
%      - Should be computed
%      - BUT: Be aware of (accidental) DOS (all web requests are hold up by long running computations)


% section evaluation end

% chapter tests-and-evaluation end
% chapter results_and_evaluation end

